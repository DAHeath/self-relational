\documentclass[p.tex]{subfiles}
\begin{document}
\section{Overview}\label{sec:overview}

In this section, we demonstrate our proof system with a small example.

\begin{figure}
\begin{lstlisting}
  s := 0;
  t := 0;
  i := 0;
  while (i < n) { // s = i(i+1)/2 & t = 0
    s := s+i;
    i := i+1;
  };
  i := 0;
  while (i < n) { // s = n(n+1)/2 & t = i(i+1)/2
    t := t+i;
    i := i+1;
  };
  assert (s = t)
\end{lstlisting}
  \caption{%
    An imperative program with two loops computing equivalent
    functions.
  }\label{fig:overview-example}
\end{figure}

Consider the small imperative program in \cref{fig:overview-example}.
The program has two loops that both compute the $n^{th}$ triangle
number (i.e. the sum from 0 to $n$). As human observers we can clearly
see that the assertion at the end of the fragment is valid: Every run
of this program results in a state where $s$ and $t$ have the same
value. The key to this observation is the similarity between the two
loops.

Unfortunately, traditional proof methods are not poised to take
advantage of the simple structure of this program. In a traditional
Hoare Logic approach, for example, the prover must find \emph{loop
invariants} which support the assertion at the end of the program.
The code fragment contains sufficient loop invariants that support the
assertion. However, these invariants rely on nonlinear arithmetic,
and thus most automated approaches will fail to find them.
%
Moreover this approach has completely missed the simplicity of the
given problem: The two loops together contain simple relationships
which support the conclusion. However, the proof methodology we have
adopted thus far forces us to consider the loops \emph{separately}.

Our approach seeks to augment the traditional proof method of Hoare
Logic by providing an elegant technique for considering various parts of the
same program \emph{simultaneously}. By carefully applying this
technique, a prover can solve proof burdens like the one specified in
\cref{fig:overview-example} more simply.
Specifically, in this example, the prover can decide to consider both
loops simultaneously.

\end{document}
