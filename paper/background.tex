\documentclass[p.tex]{subfiles}
\begin{document}
\section{Background}\label{sec:background}

\subsection{Target Language}\label{sec:language}

\begin{figure}
\begin{spreadlines}{5pt}
\begin{gather*}
  \inference[\textsc{E-Skip}]{%
  }{%
    \ceval{\cskip}{\sigma}{\sigma}
  }
  \qquad
  \inference[\textsc{E-Assign}]{%
  }{%
    \ceval{\cassign{x}{a}}{\sigma}{\subst{x}{a}{(\sigma)}}
  }
  \\
  \inference[\textsc{E-Assert}]{%
    \eeval{b}{\sigma}{\btrue}
  }{%
    \ceval{\cassert{b}}{\sigma}{\sigma}
  }
  \qquad
  \inference[\textsc{E-Sequence}]{%
    \ceval{c_0}{\sigma_0}{\sigma_1}&
    \ceval{c_1}{\sigma_1}{\sigma_2}&
  }{%
    \ceval{\cseq{c_0}{c_1}}{\sigma_0}{\sigma_2}
  }
  \\
  \inference[\textsc{E-SumLeft}]{%
    \ceval{c_0}{\sigma_0}{\sigma_1}
  }{%
    \ceval{\csum{c_0}{c_1}}{\sigma_0}{\sigma_1}
  }
  \qquad
  \inference[\textsc{E-SumRight}]{%
    \ceval{c_1}{\sigma_0}{\sigma_1}
  }{%
    \ceval{\csum{c_0}{c_1}}{\sigma_0}{\sigma_1}
  }
  \\
  \inference[\textsc{E-Loop}]{%
    \ceval{c}{\sigma_0}{\sigma_1}&
    \ceval{\cloop{c}}{\sigma_1}{\sigma_2}&
  }{%
    \ceval{\cloop{c}}{\sigma_0}{\sigma_2}
  }
  \qquad
  \inference[\textsc{E-Break}]{%
  }{%
    \ceval{\cloop{c}}{\sigma}{\sigma}
  }
\end{gather*}
\end{spreadlines}
  \caption{%
    The formal semantics of the target language $\lang$, defined in
    terms of the constituent imperative commands in \com.
    Metavariables $\sigma, \sigma_0, \ldots$ denote program states in
    \state, metavariable $x$ denotes a variable in \var, metavariables
    $c, c_0, \ldots$ denote commands in \com, $a$ denotes an
    integer-valued expression in $\aexpr$, and $b$ denotes a
    boolean-valued expression in $\bexpr$.
  }\label{fig:semantics}
\end{figure}

In this section we present the imperative language, $\lang$, that we will extend
and analyze for the remainder of this paper.
$\lang$ includes
assignments, branches, loops, and assertions.
%
$\lang$ makes use of expressions with boolean and integer
value, denoted $\bexpr$ and $\aexpr$ respectively.
%
We denote the space of program variables as \var.
%
Each program in $\lang$ is an imperative command.
We define the space of imperative commands inductively:
\begin{align*}
  \com ::=&~\cskip\\
         |&~\cassign{\var}{\aexpr}\\
         |&~\cassert{\bexpr}\\
         |&~\cseq{\com}{\com}\\
         |&~\csum{\com}{\com}\\
         |&~\cloop{\com}
\end{align*}

The command $\cseq{c_0}{c_1}$ is the command which first executes
$c_0$ and then executes $c_1$ and the command $\cskip$ is a no-op. The
command $\cassert{b}$ is a static claim that the boolean expression
$b$ evaluates to true in every possible run of the command. We define
the formal semantics of the space of commands in \cref{fig:semantics}.
%
The program semantics are defined with reference to members of the
space of program states, which we denote $\state$. A state
is a map from a variables to integer values:
%
\[\state ::= \var \rightarrow \mathbb{Z}\]

\subsection{Hoare Logic}

TODO:
\begin{itemize}
  \item Space of assertions
  \item Definition of Hoare triple
  \item Traditional system?
\end{itemize}

\end{document}
