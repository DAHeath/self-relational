\documentclass[p.tex]{subfiles}
\begin{document}
\section{Background}\label{sec:background}

\subsection{Target Language}\label{sec:language}

\begin{figure}
\begin{spreadlines}{5pt}
\begin{gather*}
  \inference[\textsc{E-Assert}]{%
    \eeval{b}{\sigma}{\btrue}
  }{%
    \ceval{\cassert{b}}{\sigma}{\sigma}
  }
  \qquad
  \inference[\textsc{E-SumLeft}]{%
    \ceval{c_0}{\sigma_0}{\sigma_1}
  }{%
    \ceval{\csum{c_0}{c_1}}{\sigma_0}{\sigma_1}
  }
  \qquad
  \inference[\textsc{E-SumRight}]{%
    \ceval{c_1}{\sigma_0}{\sigma_1}
  }{%
    \ceval{\csum{c_0}{c_1}}{\sigma_0}{\sigma_1}
  }
  \\
  \inference[\textsc{E-Assign}]{%
  }{%
    \ceval{\cassign{x}{a}}{\sigma}{\subst{x}{a}{(\sigma)}}
  }
  \qquad
  \inference[\textsc{E-Sequence}]{%
    \ceval{c_0}{\sigma_0}{\sigma_1}&
    \ceval{c_1}{\sigma_1}{\sigma_2}&
  }{%
    \ceval{\cseq{c_0}{c_1}}{\sigma_0}{\sigma_2}
  }
  \\
  \inference[\textsc{E-Skip}]{%
  }{%
    \ceval{\cskip}{\sigma}{\sigma}
  }
  \qquad
  \inference[\textsc{E-Loop}]{%
    \ceval{c}{\sigma_0}{\sigma_1}&
    \ceval{\cloop{c}}{\sigma_1}{\sigma_2}
  }{%
    \ceval{\cloop{c}}{\sigma_0}{\sigma_2}
  }
  \qquad
  \inference[\textsc{E-Break}]{%
  }{%
    \ceval{\cloop{c}}{\sigma}{\sigma}
  }
\end{gather*}
\end{spreadlines}
  \caption{%
    The formal semantics of the target language $\lang$, defined in
    terms of the constituent imperative commands in \com.
    Metavariables $\sigma, \sigma_0, \ldots$ denote program states in
    \state, metavariable $x$ denotes a variable in \var, metavariables
    $c, c_0, \ldots$ denote commands in \com, $a$ denotes an
    integer-valued expression in $\aexpr$, and $b$ denotes a
    boolean-valued expression in $\bexpr$.
  }\label{fig:semantics}
\end{figure}

In this section we present the imperative language, $\lang$, that we will extend
and analyze for the remainder of this paper.
$\lang$ includes
assignments, branches, loops, and assertions.
%
$\lang$ makes use of expressions with boolean and integer
value, denoted $\bexpr$ and $\aexpr$ respectively.
%
We denote the space of program variables as \var.
%
Each program in $\lang$ is an imperative command.
We define the space of imperative commands inductively:
\begin{align*}
  \com ::=&~\cskip\\
         |&~\cassign{\var}{\aexpr}\\
         |&~\cassert{\bexpr}\\
         |&~\cseq{\com}{\com}\\
         |&~\csum{\com}{\com}\\
         |&~\cloop{\com}
\end{align*}

The command $\cseq{c_0}{c_1}$ is the command which first executes
$c_0$ and then executes $c_1$ and the command $\cskip$ is a no-op. The
command $\cassert{b}$ is a static claim that the boolean expression
$b$ evaluates to true in every possible run of the command. We define
the formal semantics of the space of commands in \cref{fig:semantics}.
%
The program semantics are defined with reference to members of the
space of program states, which we denote $\state$. A state
is a map from a variables to integer values:
%
\[\state ::= \var \rightarrow \mathbb{Z}\]

\subsection{Hoare Logic}
\begin{figure}
\begin{spreadlines}{5pt}
\begin{gather*}
  \inference[\textsc{Skip}]{%
  }{%
    \hoare{P}{\cskip}{P}
  }
  \qquad
  \inference[\textsc{Assign}]{%
  }{%
    \hoare{\subst{x}{a}{P}}{\cassign{x}{a}}{P}
  }
  \qquad
  \inference[\textsc{Assert}]{%
  }{%
    \hoare{P}{\cassert{b}}{P \land b}
  }
  \\
  \inference[\textsc{Seq}]{%
    \hoare{P}{c_0}{R}&
    \hoare{R}{c_1}{Q}
  }{%
    \hoare{P}{\cseq{c_0}{c_1}}{Q}
  }
  \qquad
  \inference[\textsc{Sum}]{%
    \hoare{P}{c_0}{Q}&
    \hoare{P}{c_1}{Q}
  }{%
    \hoare{P}{\csum{c_0}{c_1}}{Q}
  }
  \\
  \inference[\textsc{Loop}]{%
    \hoare{P}{c}{P}
  }{%
    \hoare{P}{\cloop{c}}{P}
  }
  \qquad
  \inference[\textsc{Cons}]{%
    P \Rightarrow P'&
    \hoare{P'}{c}{Q'}&
    Q' \Rightarrow Q
  }{%
    \hoare{P}{c}{Q}
  }
  \end{gather*}
\end{spreadlines}
\caption{%
  Hoare Logic judgements over $\lang$.
  Metavariables $P, Q, R\ldots$ represent program
  assertions. Other metavariables are consistent with
  \cref{fig:semantics}
}\label{fig:base-proof-system}
\end{figure}

Hoare Logic is a framework for reasoning about imperative commands by
relating them to program \emph{assertions}.
An assertion is a proposition over a program state. Formally, we
define the space of assertions as functions from program states to
propositions.
\[ \textsf{Assert} ::= \state \rightarrow \textsf{Prop} \]
Typically, a Hoare logic system provides a set of judgements over
\emph{Hoare Triples}.
If $P, Q \in \textsf{Assert}$ are assertions and $c \in \com$ is a
command, then \hoare{P}{c}{Q} is a Hoare Triple.
%
Informally, this triple can be interpreted as stating \textbf{(1)} if
the assertion $P$ holds over some program state $\sigma$ and
\textbf{(2)} the command $c$ takes $\sigma$ to $\sigma'$, then
\textbf{(3)} the assertion $Q$ must hold over $\sigma'$.
%
This definition can be formalized as follows:
\[
  \hoare{P}{c}{Q} ::= \forall \sigma,
  \sigma'.~\ceval{c_0}{\sigma}{\sigma'} \Rightarrow P~\sigma \Rightarrow Q~\sigma'
\]
By establishing a set of sound rules, a Hoare logic system provides a
composable framework for structuring program proofs.
We provide a set of typical Hoare logic rules for $\lang$ in
\Cref{fig:base-proof-system}.



\end{document}
