\documentclass[p.tex]{subfiles}
\begin{document}
\section{Background}\label{sec:background}

\subsection{Target Language}\label{sec:language}

\begin{figure}
\begin{spreadlines}{5pt}
  \begin{gather*}
    \inference[\textsc{E-Skip}]{%
    }{%
      \ceval{\cskip}{\sigma}{\sigma}
    }
    \qquad
    \inference[\textsc{E-Assign}]{%
    }{%
      \ceval{\cassign{x}{a}}{\sigma}{\subst{x}{a}{\sigma}}
    }
    \qquad
    \inference[\textsc{E-Assert}]{%
      \eeval{b}{\sigma}{\btrue}
    }{%
      \ceval{\cassert{b}}{\sigma}{\sigma}
    }
    \\
    \inference[\textsc{E-IfTrue}]{%
      \ceval{c_0}{\sigma_0}{\sigma_1}&
      \eeval{b}{\sigma_0}{\btrue}
    }{%
      \ceval{\cif{b}{c_0}{c_1}}{\sigma_0}{\sigma_1}
    }
    \qquad
    \inference[\textsc{E-IfFalse}]{%
      \ceval{c_1}{\sigma_0}{\sigma_1}&
      \eeval{b}{\sigma_0}{\bfalse}
    }{%
      \ceval{\cif{b}{c_0}{c_1}}{\sigma_0}{\sigma_1}
    }
    \\
    \inference[\textsc{E-WhileTrue}]{%
      \ceval{c}{\sigma_0}{\sigma_1}&
      \ceval{\cwhile{b}{c}}{\sigma_1}{\sigma_2}&
      \eeval{b}{\sigma_0}{\btrue}
    }{%
      \ceval{\cwhile{b}{c}}{\sigma_0}{\sigma_2}
    }
    \\
    \inference[\textsc{E-WhileFalse}]{%
      \eeval{b}{\sigma}{\bfalse}
    }{%
      \ceval{\cwhile{b}{c}}{\sigma}{\sigma}
    }
    \qquad
    \inference[\textsc{E-Sequence}]{%
      \ceval{c_0}{\sigma_0}{\sigma_1}&
      \ceval{c_1}{\sigma_1}{\sigma_2}&
    }{%
      \ceval{\cseq{c_0}{c_1}}{\sigma_0}{\sigma_2}
    }
  \end{gather*}
  \end{spreadlines}
  \caption{%
    The formal semantics of the target language $\lang$, defined in
    terms of the constituent imperative commands in \com.
    Metavariables $\sigma, \sigma_0, \ldots$ denote program states in
    \state, metavariable $x$ denotes a variable in \var, metavariables
    $c, c_0, \ldots$ denote commands in \com, $a$ denotes an
    integer-valued expression in $\aexpr$, and $b$ denotes a
    boolean-valued expression in $\bexpr$.
  }\label{fig:semantics}
\end{figure}

In this section we present the imperative language, $\lang$, that we will extend
and analyze for the remainder of this paper.
$\lang$ includes
assignments, conditionals, loops, and assertions.
%
$\lang$ makes use of expressions with boolean and integer
value, denoted $\bexpr$ and $\aexpr$ respectively.
%
We denote the space of program variables as \var.
%
Each program in $\lang$ is an imperative command.
We define the space of imperative commands inductively:
\begin{align*}
  \com ::=&~\cskip\\
         |&~\cassign{\var}{\aexpr}\\
         |&~\cif{\bexpr}{\com}{\com}\\
         |&~\cwhile{\bexpr}{\com}\\
         |&~\cassert{\bexpr}\\
         |&~\cseq{\com}{\com}
\end{align*}

The command $\cseq{c_0}{c_1}$ is the command which first executes
$c_0$ and then executes $c_1$ and the command $\cskip$ is a no-op. The
command $\cassert{b}$ is a static claim that the boolean expression
$b$ evaluates to true in every possible run of the command. We define
the formal semantics of the space of commands in \cref{fig:semantics}.
%
The program semantics are defined with reference to members of the
space of program states, which we denote $\state$. A state
is a map from a variables to integer values:
%
\[\state ::= \var \rightarrow \mathbb{Z}\]
%
The semantics are typical an imperative language. We will extend
$\com$ in \cref{sec:product-com}.

\subsection{Hoare Logic}

TODO:
\begin{itemize}
  \item Space of assertions
  \item Definition of Hoare triple
  \item Traditional system?
\end{itemize}

\end{document}
