\documentclass[p.tex]{subfiles}
\begin{document}
\section{Technical Approach}\label{sec:approach}
\subsection{Products of Commands}\label{sec:product-com}
In order to expand the expressive power of our proof system, we extend
$\com$ with an additional construction: Product commands.
\[ \com ::= \cdots~|~\cprod{\com}{\com} \]
Informally,
a product command represents two commands whose executions act on
different program states. However, by considering them together we can
establish relationships between different programs or even different
parts of the same program.

To give semantic meaning to this new construction, we extend the
definition of the space of program states:
\[\state ::= \var \rightarrow \mathbb{Z}~|~\sprod{\state}{\state}\]
As in \cref{sec:language}, a state contains mappings from variables to integer
values. However, this new definition extends allows multiple
`parallel' states which we can consider simultaneously. The semantics
defined previously remain identical, but the originally defined
commands only make progress when the state is a singleton map. The new
parallel construction allows us to define sensible formal semantics for
product commands:
\begin{align*}
  \inference[\textsc{E-Prod}]{%
    \ceval{c_0}{\sigma_0}{\sigma_0'}&
    \ceval{c_1}{\sigma_1}{\sigma_1'}&
  }{%
    \ceval{\cprod{c_0}{c_1}}{\sprod{\sigma_0}{\sigma_1}}{\sprod{\sigma_0'}{\sigma_1'}}
  }
\end{align*}
That is, two parallel commands executed over two parallel states
execute independently.

\subsection{A Self-Relational Proof System}
\begin{figure}
\begin{spreadlines}{5pt}
\begin{gather*}
  \inference[\textsc{Prod}]{%
    \forall \sigma_1.
      \hoare{\aleft\left(P,\sigma_1\right)}
            {c_0}
            {\aleft\left(R,\sigma_1\right)}&
    \forall \sigma_0.
      \hoare{\aright\left(R,\sigma_0\right)}
            {c_1}
            {\aright\left(Q,\sigma_0\right)}
  }{%
    \hoare{P}{c_0 \times c_1}{Q}
  }
  \\
  \inference[\textsc{Dupl}]{%
    \hoare{\ajoin\left(P\right)}{\cprod{c}{c}}{\ajoin\left(Q\right)}
  }{%
    \hoare{P}{c}{Q}
  }
  \qquad
  \inference[\textsc{Iso}]{%
    c_0 \cong c_1&
    \hoare{P}{c_1}{Q}
  }{%
    \hoare{P}{c_0}{Q}
  }
\end{gather*}
\end{spreadlines}
\caption{%
  Hoare Logic rules that extend the base proof system in
  \cref{fig:base-proof-system} in order to allow reasoning about
  products of commands.
  %
  $\textsf{commute}$ and $\textsf{associate}$.
  %
  All metavariables are consistent with \cref{fig:base-proof-system}
}\label{fig:proof-rules}
\end{figure}

Now that we have extended $\lang$ with products that have sensible
semantics, we can extend the proof system of
\cref{fig:base-proof-system}.
Specifically, we would like to add rules that incorporate products.

Recall that our goal is to allow the prover to consider two parts of
the same program simultaneously. Syntactically, this goal can be
restated as wanting to replace a sequence of commands $\cseq{c_0}{c_1}$
by a product of commands $\cprod{c_0}{c_1}$ in the proof.
The following proof rule captures this goal and is our key technical
contribution:

\begin{empheq}[box=\fbox]{gather*}
  \inference[\textsc{Seq-Prod}]{%
    \hoare{\ajoin(P)}{\cskip \times c_0}{R}&
    \hoare{R}{c_0 \times c_1}{S}&
    \hoare{S}{c_1 \times \cskip}{\ajoin(Q)}
  }{%
    \hoare{P}{\cseq{c_0}{c_1}}{Q}
  }
  \\
  \ajoin(P) ::= \lambda(\sigma_0 \otimes \sigma_1).~(\sigma_0 = \sigma_1) \land P(\sigma_0)
\end{empheq}
An informal reading of this rule follows.
Suppose that we want to reason about $\hoare{P}{\cseq{c_0}{c_1}}{Q}$ by
considering $c_0$ and $c_1$ together. First, we must establish a
``mutual'' pre-condition $R$. This pre-condition relates the state
before executing $c_0$ to the state after executing $c_0$ (but before
executing $c_1$).
We must demonstrate that starting from $P$ and
executing $c_0$ leads to this mutual pre-condition
    (first premise).
Symmetrically, we must establish a mutual post-condition $S$. $S$
relates the state after executing $c_0$ (but before executing $c_1$)
to the state after executing both commands.
We must demonstrate that starting from this mutual post-condition and
executing $c_1$ leads to the post-condition $Q$ (third premise).
Finally, we have achieved our goal: After showing the first and third
premise, it suffices to show that the product $\cprod{c_0}{c_1}$ with
pre-condition $R$ leads to post-condition $S$ (second premise).

\textsc{Seq-Prod} is the key rule for giving our proof system
expressive power, but it is not sufficient to stop here.
\Cref{fig:proof-rules} provides additional rules needed to dispatch
relational proofs.
%
\textsc{Iso} is a helpful rule that allows the prover to rewrite the
target command to some other, isomorphic command. We will discuss
some useful isomorphisms in \cref{sec:iso}.
%
\textsc{Comm} and \textsc{Assoc} relate to dealing with multiple
nested products.

\subsection{Command Isomorphisms}\label{sec:iso}

\begin{figure}
\begin{spreadlines}{5pt}
\begin{gather*}
  \inference[$\textsc{I-Skip}_L$]{%
  }{%
    \cseq{\cskip}{c} \cong c
  }
  \qquad
  \inference[$\textsc{I-Skip}_R$]{%
  }{%
    \cseq{c}{\cskip} \cong c
  }
  \qquad
  \inference[$\textsc{I-SeqAssoc}$]{%
  }{%
    \cseq{\left(\cseq{c_0}{c_1}\right)}{c_2}
    \cong
    \cseq{c_0}{\left(\cseq{c_1}{c_2}\right)}
  }
  \\
  \inference[\textsc{I-SeqProd}]{%
  }{%
    \cprod{\left(\cseq{c_0}{c_1}\right)}
          {\left(\cseq{c_2}{c_3}\right)} \cong
    \cseq{\left(\cprod{c_0}{c_2}\right)}
         {\left(\cprod{c_1}{c_3}\right)}
  }
  \qquad
  \inference[\textsc{I-SumProd}]{%
  }{%
    \cprod{(\csum{c_0}{c_1})}{c_2} \cong
    \csum{(\cprod{c_0}{c_2})}{(\cprod{c_1}{c_2})}
  }
  \\
  \inference[\textsc{I-Unroll}]{%
    \text{TODO legal based on context of $c$}
  }{%
    \cloop{c} \cong \cloop{\cseq{c}{c}}
  }
  \qquad
  \inference[\textsc{I-LoopProd}]{%
    \text{TODO legal based on context of $c_0, c_1$}
  }{%
    \cprod{\cloop{c_0}}{\cloop{c_1}} \cong \cloop{\cprod{c_0}{c_1}}
  }
\end{gather*}
\end{spreadlines}
\caption{%
  Program isomorphisms that are useful in conjunction with the
  self-relational proof system. Metavariables are consistent with
  \cref{fig:semantics}.
}
\end{figure}

\subsection{An Inference Algorithm}

\begin{figure}
\begin{lstlisting}{haskell}
{} _ {} :: Assertion -> Com -> Assertion -> ()
{p} c {q}
  case reassociate c of
    c0 ; c1 ->
      if loopless c0 || loopless c1
      then do
        r <- freshPredicate
        {p} c0 {r}
        {r} c1 {q}
      else do
        r <- freshPredicate
        s <- freshPredicate
        {join p} Skip * c0 {r}
        {r} c0 * c1 {s}
        {s} c1 * Skip {q * q}
    c0 + c1 -> do
      {p} c0 {q}
      {p} c1 {q}
    Loop c -> do
      p ==> q
      triple {q} c {q}
    c0 * c1 ->
      if loopless c0 || loopless c1
      then do
        st0 * st1 <- freshState
        r <- freshPredicate
        quantifyRight st1 ({p} c0 {r})
        quantifyLeft st0 ({r} c1 {q})
      else {p} rewrite c0 c1 {q}
\end{lstlisting}
\end{figure}

\end{document}
