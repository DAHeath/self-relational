\documentclass{article}

\usepackage{amsmath}
\usepackage{semantic}
\usepackage{mathtools}
\usepackage{fullpage}

\newcommand{\triple}[3]{\{#1\}~#2~\{#3\}}
\newcommand{\ppart}[3]{#1~\textsf{part}~#2\cdot#3}
\newcommand{\judges}{\vdash}
\newcommand{\judgement}[4]{#1 \judges \triple{#2}{#3}{#4}}

\newcommand{\cif}[3]{\textsf{if}~#1~\textsf{then}~#2~\textsf{else}~#3~\textsf{fi}}
\newcommand{\cwhile}[2]{\textsf{while}~#1~\textsf{do}~#2~\textsf{done}}
\newcommand{\cskip}{\textsf{skip}}

\newcommand{\commute}{\textsf{commute}}
\newcommand{\associate}{\textsf{associate}}
\newcommand{\pairwise}{\textsf{pairwise}}

\begin{document}

The space of states is defined inductively as follows:

\[ \sigma ::= \sigma \otimes \sigma~|~x \rightarrow \nu \]

That is, a state is either a product of two states, or is a mapping from
variables to values.

An assertion ($P$, $Q$, etc) is a mapping from states to propositions,
as in traditional Hoare Logic:
\[P : \sigma \rightarrow \textsf{Prop}\]

\begin{spreadlines}{10pt}
\begin{gather*}
  \inference[\textsc{Part}]{%
    \ppart{c_0}{c_1}{c_2}&
    \judgement{\Gamma}{P}{c_1}{R}\\
    \judgement{\Gamma}{\pairwise~P~R}{c_1 \times c_2}{\pairwise~R~Q}
  }{%
    \judgement{\Gamma}{P}{c_0}{Q}
  }
  \\
  \inference[\textsc{Split}]{%
    \judgement{\Gamma}{P_0}{c_0}{Q_0}&
    \forall st_0, st_1.~P~(st_0 \otimes st_1) \Rightarrow P_0~st_0 \land P_1~st_1\\
    \judgement{\Gamma}{P_1}{c_1}{Q_1}&
    \forall st_0, st_1.~Q~(st_0 \otimes st_1) \Leftarrow Q_0\!~st_0 \land Q_1\!~st_1\\
  }{%
    \judgement{\Gamma}{P}{c_0 \times c_1}{Q}
  }
  \\
  \inference[\textsc{Comm}]{%
    \judgement{\Gamma}{\commute~P}{c_1 \times c_0}{\commute~Q}
  }{%
    \judgement{\Gamma}{P}{c_0 \times c_1}{Q}
  }
  \qquad
  \inference[\textsc{Assoc}]{%
    \judgement{\Gamma}{\associate~P}{(c_0 \times c_1) \times c_2}{\associate~Q}
  }{%
    \judgement{\Gamma}{P}{c_0 \times (c_1 \times c_2)}{Q}
  }
  \\
  \inference[$\textsc{If}$]{%
    \judgement{\Gamma}{P \land ~~e}{c_0 \times c_2}{Q}\\
    \judgement{\Gamma}{P \land \neg e}{c_1 \times c_2}{Q}
  }{%
    \judgement{\Gamma}{P}{\cif{e}{c_0}{c_1} \times c_2}{Q}
  }
  \\
  \inference[$\textsc{While}$]{%
    \judgement{%
      \Gamma \cup \triple{P}{\cwhile{e}{c_0} \times c_1}{Q \land \neg e}
    }{P}{\cif{e}{c_0}{\cskip}~;~\cwhile{e}{c_0} \times c_1}{Q}
  }{%
    \judgement{\Gamma}{P}{\cwhile{e}{c_0} \times c_1}{Q \land \neg e}
  }
\end{gather*}
\end{spreadlines}

\begin{align*}
  \pairwise~P~Q~(\sigma_0\otimes\sigma_1) &:= P~\sigma_0 \land Q~\sigma_1\\
  \commute~P~(\sigma_0 \otimes \sigma_1) &:= P~(\sigma_1 \otimes \sigma_0)\\
  \associate~P~((\sigma_0 \otimes \sigma_1) \otimes \sigma_2)) &:=
    P~(\sigma_0 \otimes (\sigma_1 \otimes \sigma_2))
\end{align*}

\begin{spreadlines}{10pt}
\begin{gather*}
  \inference[E-Prod]{%
    (c_0, st_0) \Downarrow (st_0')&
    (c_1, st_1) \Downarrow (st_1')
  }{%
    (c_0 \times c_1, st_0 \otimes st_1) \Downarrow (st_0' \otimes st_1')
  }
\end{gather*}
\end{spreadlines}

A more typical sequencing rule can be derived from $\textsc{Part}$ and
$\textsc{Split}$:

\[
  \inference[\textsc{Seq}]{%
    \ppart{c_0}{c_1}{c_2}&
    \judgement{\Gamma}{P}{c_1}{R}&
    \judgement{\Gamma}{R}{c_2}{Q}
  }{%
    \judgement{\Gamma}{P}{c_0}{Q}
  }
\]


\end{document}
